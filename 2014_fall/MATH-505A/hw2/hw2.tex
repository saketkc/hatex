%%%%%%%%%%%%%%%%%%%%%%%%%%%%%%%%%%%%%%%%%
% Structured General Purpose Assignment
% LaTeX Template
%
% This template has been downloaded from:
% http://www.latextemplates.com
%
% Original author:
% Ted Pavlic (http://www.tedpavlic.com)
%
% Note:
% The \lipsum[#] commands throughout this template generate dummy text
% to fill the template out. These commands should all be removed when 
% writing assignment content.
%
%%%%%%%%%%%%%%%%%%%%%%%%%%%%%%%%%%%%%%%%%

%----------------------------------------------------------------------------------------
%	PACKAGES AND OTHER DOCUMENT CONFIGURATIONS
%----------------------------------------------------------------------------------------

\documentclass{article}

\usepackage{fancyhdr} % Required for custom headers
\usepackage{lastpage} % Required to determine the last page for the footer
\usepackage{extramarks} % Required for headers and footers
\usepackage{graphicx} % Required to insert images
\usepackage{lipsum} % Used for inserting dummy 'Lorem ipsum' text into the template

\usepackage{amsmath}
%\usepackage{multline}

% Margins
\topmargin=-0.45in
\evensidemargin=0in
\oddsidemargin=0in
\textwidth=6.5in
\textheight=9.0in
\headsep=0.25in 

\linespread{1.1} % Line spacing

% Set up the header and footer
\pagestyle{fancy}
\lhead{\hmwkAuthorName} % Top left header
\chead{\hmwkClass\ : \hmwkTitle} % Top center header
\rhead{\firstxmark} % Top right header
\lfoot{\lastxmark} % Bottom left footer
\cfoot{} % Bottom center footer
\rfoot{Page\ \thepage\ of\ \pageref{LastPage}} % Bottom right footer
\renewcommand\headrulewidth{0.4pt} % Size of the header rule
\renewcommand\footrulewidth{0.4pt} % Size of the footer rule

\setlength\parindent{0pt} % Removes all indentation from paragraphs

%----------------------------------------------------------------------------------------
%	DOCUMENT STRUCTURE COMMANDS
%	Skip this unless you know what you're doing
%----------------------------------------------------------------------------------------

% Header and footer for when a page split occurs within a problem environment
\newcommand{\enterProblemHeader}[1]{
\nobreak\extramarks{#1}{#1 continued on next page\ldots}\nobreak
\nobreak\extramarks{#1 (continued)}{#1 continued on next page\ldots}\nobreak
}

% Header and footer for when a page split occurs between problem environments
\newcommand{\exitProblemHeader}[1]{
\nobreak\extramarks{#1 (continued)}{#1 continued on next page\ldots}\nobreak
\nobreak\extramarks{#1}{}\nobreak
}

\setcounter{secnumdepth}{0} % Removes default section numbers
\newcounter{homeworkProblemCounter} % Creates a counter to keep track of the number of problems

\newcommand{\homeworkProblemName}{}
\newenvironment{homeworkProblem}[1][Problem \arabic{homeworkProblemCounter}]{ % Makes a new environment called homeworkProblem which takes 1 argument (custom name) but the default is "Problem #"
\stepcounter{homeworkProblemCounter} % Increase counter for number of problems
\renewcommand{\homeworkProblemName}{#1} % Assign \homeworkProblemName the name of the problem
\section{\homeworkProblemName} % Make a section in the document with the custom problem count
\enterProblemHeader{\homeworkProblemName} % Header and footer within the environment
}{
\exitProblemHeader{\homeworkProblemName} % Header and footer after the environment
}

\newcommand{\problemAnswer}[1]{ % Defines the problem answer command with the content as the only argument
\noindent\framebox[\columnwidth][c]{\begin{minipage}{0.98\columnwidth}#1\end{minipage}} % Makes the box around the problem answer and puts the content inside
}

\newcommand{\homeworkSectionName}{}
\newenvironment{homeworkSection}[1]{ % New environment for sections within homework problems, takes 1 argument - the name of the section
\renewcommand{\homeworkSectionName}{#1} % Assign \homeworkSectionName to the name of the section from the environment argument
\subsection{\homeworkSectionName} % Make a subsection with the custom name of the subsection
\enterProblemHeader{\homeworkProblemName\ [\homeworkSectionName]} % Header and footer within the environment
}{
\enterProblemHeader{\homeworkProblemName} % Header and footer after the environment
}
   
%----------------------------------------------------------------------------------------
%	NAME AND CLASS SECTION
%----------------------------------------------------------------------------------------

\newcommand{\hmwkTitle}{Homework\ \# 2 } % Assignment title
\newcommand{\hmwkDueDate}{Friday,\ September \ 5,\ 2014} % Due date
\newcommand{\hmwkClass}{MATH-505A} % Course/class
\newcommand{\hmwkClassTime}{10:30am} % Class/lecture time
\newcommand{\hmwkAuthorName}{Saket Choudhary} % Your name
\newcommand{\hmwkAuthorID}{2170058637} % Teacher/lecturer
%----------------------------------------------------------------------------------------
%	TITLE PAGE
%----------------------------------------------------------------------------------------

\title{
\vspace{2in}
\textmd{\textbf{\hmwkClass:\ \hmwkTitle}}\\
\normalsize\vspace{0.1in}\small{Due\ on\ \hmwkDueDate}\\
%\vspace{0.1in}\large{\textit{\hmwkClassTime}}
\vspace{3in}
}

\author{\textbf{\hmwkAuthorName} \\
	\textbf{\hmwkAuthorID}
	}
\date{} % Insert date here if you want it to appear below your name

%----------------------------------------------------------------------------------------

\begin{document}

\maketitle

%----------------------------------------------------------------------------------------
%	TABLE OF CONTENTS
%----------------------------------------------------------------------------------------

%\setcounter{tocdepth}{1} % Uncomment this line if you don't want subsections listed in the ToC

\newpage
\tableofcontents
\newpage

\begin{homeworkProblem}[Exercise \# 1.5] % Custom section title


	\begin{homeworkSection}{(1)} % Section within problem
		
		\problemAnswer{ % Answer
				\textbf{Given:} $A,B$ are independent \\
				\textbf{To Prove:} $(A^C,B)$; $(A^C, B^C)$ are independent \\
				
				Since $A,B$ are independent: \\
				\begin{equation}
					\label{1c1}
					P(A \cap B) = P(A)(B)				
				\end{equation}	
				Thus,
				\begin{equation}
					\label{1c2}
					P(A \cap B) = (1-P(A^C))P(B)  = P(B) - P(B)P(A^C)
				\end{equation}
				
				Rearranging \ref{1c2}:
				\begin{equation}
					\label{1c3}
					P(B)P(A^C) = P(B) - P(A\cap B) 
				\end{equation}
				
				$P(B)-P(A \cap B)$ signifies \textbf{'in $B$ but not in $A$ $AND$ $B'$.} Thus, it should belong to $A^C $ $AND$ $B$ 
				\begin{equation}
					\label{1c4}
					P(B) - P(A \cap B) = P(A^C \cap B) = P(A^C)P(B)
				\end{equation}
				From \ref{1c3} and \ref{1c4} : $A^C.B $ are independent.
				
				Similiary to prove $A^C, B^C$ are independent, we perform substitute $B^C$ in $P(B \cap A^C)$
			}
	
	\end{homeworkSection}
	\begin{homeworkSection}{(2)} % Section within problem
		
		\problemAnswer{ % Answer
			$A_{ij} = i^{th}$ and $j^{th}$ rolls produce the same number.\\
			For any $i \neq j$, total outcome are $6*6=36$ and number  of favourable outcomes are $\binom{6}{1} * 1 = 6$, thus \\
			$p(A_{ij}) = \frac{6}{36} = \frac{1}{6} $ \\
			Consider $P(A_{ij} \cap A_{kj})$, such that $i \neq j \neq k$, then : \\
			\center $P(A_ij \cap A_jk) $ refers to the probability when $i^{th}, j^{th}$ and $k^{th}$ rolls show the same number, which can be caluclated as:
			$P(A_{ij} \cap A_{kj}) = \frac{\binom{6}{1} * 1 * 1}{6 * 6 * 6}= \frac{1}{36} = P(A_{ij})P(A_{kj}) $
				
				Thus, $A_{ij}$ are pairwise independent as it is true for any choice of $i,j, k$ as long as $i \neq j \neq k$
				
				Consider:
				
			\center	$P(A_ij \cap A_jk \cap A_kl ) =  \frac{\binom{6}{1} * 1 * 1 * 1}{6*6*6} = \frac{1}{36} \neq P(A_{ij}) P(A_{jk}) P(A_{kl}) $
				
				Since 	$P(A_ij \cap A_jk \cap A_kl ) \neq P(A_{ij}) P(A_{jk})P(A_{kl})$, it will not be true in general.
				
				And since the independence criterion is not satisfied for the above case, it will not be true for a case for all $A_{ij}$ are considered togethter for all values of $i,j$.
			}
			
			
		\end{homeworkSection}
		
			\begin{homeworkSection}{(3)} 
				
				\problemAnswer{ % Answer
					\textbf{To Prove: }
					
					(a) outcomes of coin tosses are independent \\
					(b) Given a sequence of length m of heads and tails the chance of it occuring in first m tosses i s $2^-m$.\\
					
					In order to prove (a) and (b) are equivalent it is sufficient to prove that if $a \implies b$ and $b \implies a$. 
					
					If the outcomes are independent, probability of a head or tail in a sequence is $\frac{1}{2}$. Consider $m$ tosses, since they are independent the probability of seeing any string of H and T is given by $\frac{1}{2}*\frac{1}{2}*...*(m )times = \frac{1}{2^m} = 2^{-m}$. Hence $\ \implies b$.
					
					Now consider if $b$ is true, then : P(m) = $2^{-m}$ $\implies P(m+1) = 2^{-(m+1)}$, . The $P(m+1)$ case is similar to $P(m)$ with an extra toss. $P(m+1)=2^{-m} * \frac{1}{2}$ . The extra half factor is accounted by the extra toss that is performed which must be independent wit respect to the m tosses for yielding such a relation for $P(m+1)$ $\implies$ $a$ is true
					
					and hence $a \impliedby b$ and $a \implies b$		
				}
				
			\end{homeworkSection}
			\begin{homeworkSection}{(4)} % Section within problem
				
				\problemAnswer{ 
					\textbf{Given:} $\Omega = \{1,2,3,...p\}$ where $p$ is prime. $F$ is set of all subsets of $\omega$; $P(A)=\frac{|A|}{p}$
					\textbf{To Prove:} $A$ 'or' $B$ is a null set or is the set $Omega$ 
					
					\begin{math}
						P(A) = \frac{|A|}{p}
					\end{math}
					
					\begin{math}
						P(B) = \frac{|B|}{p}
					\end{math}
					
					Now, by definition: \\
					\begin{equation}
					\label{4c1}
					P(A \cap B) = \frac{|A \cap B|}{p}
					\end{equation}
					
					Since $A,B$ are independent: \\
					\begin{equation}
					\label{4c2}
					P(A \cap B) = P(A)P(B) = \frac{|A \cap B|}{p} = \frac{|A|}{p}\frac{|B|}{p}
					\end{equation}
					
					
					Thus: \\
					\begin{equation}
					\label{4c3}
					p |A \cap B | = |A||B| \implies
					|A||B|\  mod\ p = 0
					\end{equation} where $mod$ operator gives the remainder.
					
					and 
					\center $ 0 \leq  |A|, |B| \leq p $ $\implies$ $|A| or |B| = p OR |A|, |B| =0$ $\implies$ A,B are either null or complet sets(with $|A|,|B|=|\Omega|$).
					
				}
				
			\end{homeworkSection}	
			\begin{homeworkSection}{(5)} % Section within problem
				
				\problemAnswer{ % Answer

					
					\textbf{	Given: 
						\begin{equation}
						\label{5c1}
						P(A, B |C) = P(A|C)P(B|C)
						\end{equation}
						  for all A,B.}
					For which event C, are A and B(for all A,B)  independent \textbf{iff} they are conditionaly independent given C: $?$
					
					If $A,B$ are independent : 
					\begin{equation}
						\label{5c2}
						P(A|B)=P(A)
					\end{equation}
					
					\begin{equation}
						\label{5c3}
						P(A,B | C) = \frac{P(A, B,C)}{P(C)} = P(A | B,C)*\frac{P(B,C)}{P(C)} = {P(A | B,C)}*{P(B|C)} 
					\end{equation}
					
					Comparing \ref{5c1} and \ref{5c3}, we need to prove that $P(A|C)=P(A|B,C)$ given that $P(A|B)=P(A)$
					For $P(A|B) =P(A)$ can be made true if we set P(C)$P(A, B |C) = P(A|C)P(B|C)$
					
					\center \textbf{Thus $P(C)=1$}
				}
				
			\end{homeworkSection}
			%\begin{homeworkSection}{(6)} % Section within problem
				
			%	\problemAnswer{ % Answer
					
			%	}
				
			%\end{homeworkSection}	
			\begin{homeworkSection}{(7)} % Section within problem
				
				\problemAnswer{ % Answer
					$A$ = $\{$all children of same sex $\}$ \\
					$B$ = $\{$ there is at most one boy $\}$ \\
					$C$ = $\{$ one boy and one girl included $\}$ \\
					
					$P(A) = P($ all boys $) + P($all girls$)$ $=\frac{1}{2}*\frac{1}{2}*\frac{1}{2}*2 = \frac{1}{4}$ \\
					
					$P(B) = P($ 0 boys $)+P($+$1 boy)$ =  $\frac{1}{8} + 3*\frac{1}{8} = \frac{1}{2}$ \\
					
					$P(C) = P($1 boy + 1 girl  $)= 2*\frac{1}{2} * ]frac{1}{2}*\frac{1}{2}*3 = \frac{3}{4}$\\
					
					\textbf{Part a): } A is independent of B and B is independent of C
					
					$P(A \cap B) $ = all chidren are of same sex AND there is at most one boy $\implies$  all children are boys \\
					$\implies$ $P(A \cap B) = \frac{1}{2} * \frac{1}{2} * \frac{1}{2} = \frac{1}{8} = P(A) * P(B) $.
					
					Hence A,B are independent \\
					
					$P(B \cap C)$ = there is at most one boy AND there is one boy and a girl $\implies$ there is one boy and two girls.
					$P(B \cap C) = \frac{1}{2} * \frac{1}{2} * \frac{1}{2} *3 = \frac{3}{8} = P(B)* P(C)$
					
					Hence B,C are independent \\
					
					\textbf{Part b): } Is A independednt of C?
					
					$P(A \cap C)$ = the family includes boy and girl AND all children are of same sex \\
					
					Clearky $P(A \cap B) = \phi$ and hence A,B are not necessarily independent!
					
					\textbf{Part c): } Do the results hold if boys and girls are not equally likely? 
					
					NO. Since
					
					\textbf{Part d):} Do these results hold if there are 4 children?
					
					Yes, the calculaions are independent of the number of children since independence relations are not dependent on the number.
			}
				
			\end{homeworkSection}			
\end{homeworkProblem}

\end{document}
