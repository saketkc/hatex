%===============================================================================
% $Id: ifacconf.tex 19 2011-10-27 09:32:13Z jpuente $  
% Template for IFAC meeting papers
% Copyright (c) 2007-2008 International Federation of Automatic Control
%===============================================================================
\documentclass{ifacconf}

%\input{./macros.tex}
\usepackage{graphicx}      % include this line if your document contains figures
\usepackage{natbib}        % required for bibliography
\usepackage{amsmath, amsxtra, amsfonts,amscd,amssymb}
\setcounter{tocdepth}{3}
\usepackage{graphicx,wrapfig}
\usepackage{epstopdf}
\usepackage{url}
\usepackage[algo2e,linesnumbered, vlined,ruled]{algorithm2e}
\usepackage{float}
\usepackage{multirow}
\usepackage{mathrsfs}
%===============================================================================

\begin{document}
\begin{frontmatter}


\title{
	Linear Mixed-Effect Models}% \thanksref{footnoteinfo}}    % Title, preferably not more than 10 words.

%\thanks[footnoteinfo]{}

\author[First]{Saket Choudhary} 




\address[First]{University of Southern California, 
   LA, CA 90089 USA (e-mail: skchoudh@ usc.edu).}


\begin{abstract}                % Abstract of not more than 250 words.
\emph{Linear Mixed-Effect Models} are an extension of \emph{Linear Regression} that describe the relationship between response variable $\mathcal{Y}$ and independent variables ${X}$ such that the coefficients can vary with respect to one or more grouping variables and hence at least one independent covariate should be categorical.

Mixed-Effects models find use in \emph{longitudinal} or repeated measures study, where repeated measurements are made on \emph{experimental} or \emph{observational} units.

Mixed-Effects models make use of constrained optimisation to arrive at the Maximum Likelihood or Restricted Maximum Likelihood estimate of the parameters.

\end{abstract}


\end{frontmatter}

\section{Problem description}
Consider a Linear Regression Problem:
$$
\mathcal{Y} = X\beta + \epsilon
$$

Where $\epsilon \sim \mathcal{N}(0,\sigma^2I)$
and $\beta$ is a p-dimensional coefficient vector;
$X$ is $n \times p$ model matrix. There are two parameters
in this model: $\beta$ and $\sigma^2$

and hence for a linear model:
$$
y \sim \mathcal{N}({X\beta,\sigma^2I})
$$


Mixed-effects models the response with an additional "random-effect" $\mathcal{B}$ such that:

$$
(\mathcal{Y}|\mathcal{B}=b) = \mathcal{N}(XB+Zb, \sigma^2I)
$$
where $Z$ is a $n \times q$ model matrix just like $X$ but for the random-effect covariates $\mathcal{B}$ which we fix at $b$ and then model $b$ as another normal random variable:

$$
\mathcal{B} \sim \mathcal{N}(0,\Sigma)
$$

where $\Sigma$ is a parameterized $q\times q$ covariance matrix.
The parameter estimation now can be down by separating(profiling) the log-likelihood.(Details Skipped, since I do not understand them yet)




\section{Goals}

\begin{itemize}
	\item Understand the derivation/math behind parameter estimation
	\item Use available modeling libraries to demonstrate at least one use case  of mixed-effects models
\end{itemize}


\section{References}
\begin{itemize}
\item Pinherio, J. C., and D. M. Bates. Mixed-Effects Models in S and S-PLUS. Statistics and Computing Series, Springer, 2004.

\item Bates, Douglas, et al. "Fitting linear mixed-effects models using lme4." arXiv preprint arXiv:1406.5823 (2014).
	
\end{itemize}



\end{document}
