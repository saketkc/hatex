%%%%%%%%%%%%%%%%%%%%%%%%%%%%%%%%%%%%%%%%%
% Structured General Purpose Assignment
% LaTeX Template
%
% This template has been downloaded from:
% http://www.latextemplates.com
%
% Original author:
% Ted Pavlic (http://www.tedpavlic.com)
%
% Note:
% The \lipsum[#] commands throughout this template generate dummy text
% to fill the template out. These commands should all be removed when 
% writing assignment content.
%
%%%%%%%%%%%%%%%%%%%%%%%%%%%%%%%%%%%%%%%%%

%----------------------------------------------------------------------------------------
%	PACKAGES AND OTHER DOCUMENT CONFIGURATIONS
%----------------------------------------------------------------------------------------

\documentclass{article}

\usepackage{fancyhdr} % Required for custom headers
\usepackage{lastpage} % Required to determine the last page for the footer
\usepackage{extramarks} % Required for headers and footers
\usepackage{graphicx} % Required to insert images
\usepackage{latexsym}
\usepackage{mathtools}

\usepackage{lipsum} % Used for inserting dummy 'Lorem ipsum' text into the template

\usepackage{amsmath}
%\usepackage{multline}

% Margins
\topmargin=-0.45in
\evensidemargin=0in
\oddsidemargin=0in
\textwidth=6.5in
\textheight=9.0in
\headsep=0.25in 

\linespread{1.1} % Line spacing

% Set up the header and footer
\pagestyle{fancy}
\lhead{\hmwkAuthorName} % Top left header
\chead{\hmwkClass\ : \hmwkTitle} % Top center header
\rhead{\firstxmark} % Top right header
\lfoot{\lastxmark} % Bottom left footer
\cfoot{} % Bottom center footer
\rfoot{Page\ \thepage\ of\ \pageref{LastPage}} % Bottom right footer
\renewcommand\headrulewidth{0.4pt} % Size of the header rule
\renewcommand\footrulewidth{0.4pt} % Size of the footer rule

\setlength\parindent{0pt} % Removes all indentation from paragraphs

%----------------------------------------------------------------------------------------
%	DOCUMENT STRUCTURE COMMANDS
%	Skip this unless you know what you're doing
%----------------------------------------------------------------------------------------

% Header and footer for when a page split occurs within a problem environment
\newcommand{\enterProblemHeader}[1]{
\nobreak\extramarks{#1}{#1 continued on next page\ldots}\nobreak
\nobreak\extramarks{#1 (continued)}{#1 continued on next page\ldots}\nobreak
}

% Header and footer for when a page split occurs between problem environments
\newcommand{\exitProblemHeader}[1]{
\nobreak\extramarks{#1 (continued)}{#1 continued on next page\ldots}\nobreak
\nobreak\extramarks{#1}{}\nobreak
}

\setcounter{secnumdepth}{0} % Removes default section numbers
\newcounter{homeworkProblemCounter} % Creates a counter to keep track of the number of problems

\newcommand{\homeworkProblemName}{}
\newenvironment{homeworkProblem}[1][Problem \arabic{homeworkProblemCounter}]{ % Makes a new environment called homeworkProblem which takes 1 argument (custom name) but the default is "Problem #"
\stepcounter{homeworkProblemCounter} % Increase counter for number of problems
\renewcommand{\homeworkProblemName}{#1} % Assign \homeworkProblemName the name of the problem
\section{\homeworkProblemName} % Make a section in the document with the custom problem count
\enterProblemHeader{\homeworkProblemName} % Header and footer within the environment
}{
\exitProblemHeader{\homeworkProblemName} % Header and footer after the environment
}

\newcommand{\problemAnswer}[1]{ % Defines the problem answer command with the content as the only argument
\noindent\framebox[\columnwidth][c]{\begin{minipage}{0.98\columnwidth}#1\end{minipage}} % Makes the box around the problem answer and puts the content inside
}

\newcommand{\homeworkSectionName}{}
\newenvironment{homeworkSection}[1]{ % New environment for sections within homework problems, takes 1 argument - the name of the section
\renewcommand{\homeworkSectionName}{#1} % Assign \homeworkSectionName to the name of the section from the environment argument
\subsection{\homeworkSectionName} % Make a subsection with the custom name of the subsection
\enterProblemHeader{\homeworkProblemName\ [\homeworkSectionName]} % Header and footer within the environment
}{
\enterProblemHeader{\homeworkProblemName} % Header and footer after the environment
}
   
%----------------------------------------------------------------------------------------
%	NAME AND CLASS SECTION
%----------------------------------------------------------------------------------------
\DeclarePairedDelimiter\ceil{\lceil}{\rceil}
\DeclarePairedDelimiter\floor{\lfloor}{\rfloor}
\newcommand{\hmwkTitle}{Homework\ \# 1 } % Assignment title
\newcommand{\hmwkDueDate}{Wednesday,\ February \ 11,\ 2015} % Due date
\newcommand{\hmwkClass}{MATH-501} % Course/class
\newcommand{\hmwkClassTime}{11:00am} % Class/lecture time
\newcommand{\hmwkAuthorName}{Saket Choudhary} % Your name
\newcommand{\hmwkAuthorID}{2170058637} % Teacher/lecturer
%----------------------------------------------------------------------------------------
%	TITLE PAGE
%----------------------------------------------------------------------------------------

\title{
\vspace{2in}
\textmd{\textbf{\hmwkClass:\ \hmwkTitle}}\\
\normalsize\vspace{0.1in}\small{Due\ on\ \hmwkDueDate}\\
%\vspace{0.1in}\large{\textit{\hmwkClassTime}}
\vspace{3in}
}

\author{\textbf{\hmwkAuthorName} \\
	\textbf{\hmwkAuthorID}
	}
\date{} % Insert date here if you want it to appear below your name

%----------------------------------------------------------------------------------------

\begin{document}

\maketitle

%----------------------------------------------------------------------------------------
%	TABLE OF CONTENTS
%----------------------------------------------------------------------------------------

%\setcounter{tocdepth}{1} % Uncomment this line if you don't want subsections listed in the ToC

\newpage
\tableofcontents
\newpage




\begin{homeworkSection}{Problem \# 1} % Section within problem

\begin{homeworkSection}{1a}
\problemAnswer{
	$f(x)=atan(x)$ on interval $[a,b] = [-4.9,5.1]$
	
	$f(a)=atan(-4.9) = 0.7854$ and $f(b) = atan(5.1)=-1.3695$
	
	Since $atan(x)\ \epsilon\ C[-4.9, 5.1]$ and $f(-4.9)f(5.1)<0$ the conditions required for bisection method
	to converge are satisfied.
	
	
	The Number of iterations is given by $M = \ceil*{log_2(\frac{b-a}{2\delta})}$ where $\delta$ = Absolute error = $10^-2$
	
	Hence $M=\ceil*{log_2(\frac{10}{2*10^-2})} = 9$
	
	
	}	
\end{homeworkSection}

\begin{homeworkSection}{1b}
\problemAnswer{
    f(-4.9) = -1.369\\
    f(5,1) = 1.377\\
	$c_0=\frac{a+b}{2} =  0.1$ and $f(c_0)=0.0997$ $\implies f(c_0)f(a)<0$
	Hence $c_1=\frac{c_0+a}{2}=-2.4$ and $f(c_1)=-1.176$ $\implies f(c_1)f(c_0)<0$\\
	Hence $c_2=\frac{c_1+c_0}{2}=-1.15$ and $f(c_2)=-0.855$ $\implies$ $f(c_2)f(c_0)<0$ so\\
	$c_3=\frac{c_2+c_0}{2}=-0.525$ and $f(c_3)=-0.05254$ and $f(c_3)f(c_0)<0$
	
	}	
\end{homeworkSection}


\begin{homeworkSection}{1c}
\problemAnswer{
\begin{tabular}{c |c| c| c}
	 \rule[-2ex]{0pt}{5.5ex} $\epsilon$ & Number of iteration & Solution  & k=$\ceil*{log_2(\frac{b-a}{2\delta})}$  \\ 
	 \rule[-2ex]{0pt}{5.5ex} $10^{-2}$ & 9  & 0.00234375  & 9 \\ 
	 \rule[-2ex]{0pt}{5.5ex} $10^{-4}$  & 12   &  -9.76562500003553e-05 & 12 \\ 
	 \rule[-2ex]{0pt}{5.5ex} $10^{-8}$  & 22 & 9.53674312853536e-08 & 22  \\ 
	 \rule[-2ex]{0pt}{5.5ex} $10^{-16}$ & 52 & -4.44089209850063e-16  & 52  \\ 
	 \rule[-2ex]{0pt}{5.5ex} $10^{-32}$  & 104  &  -9.86076131526265e-32 & 104 \\ 
	 \rule[-2ex]{0pt}{5.5ex} $10^{-64}$  & 212  &  -3.03858167864314e-64 & 212  \\ 
	 \rule[-2ex]{0pt}{5.5ex} $10^{-128}$  & 424  & -4.61648930889287e-128  & 424 \\  
	\hline 
\end{tabular} 
	
	}	

\end{homeworkSection}


\end{homeworkSection}

\begin{homeworkSection}{2}
\begin{homeworkSection}{2a}
	\problemAnswer{
		$x0 = 5$
		
		Iteration : 1 ||  x2 = 3.6266 || x1 = 2.32486 \\
		Iteration : 2 ||  x2 = 1.16027 || x1 = 3.6266  \\
		Iteration : 3 ||  x2 = 2.32486 || x1 = 1.16027 \\
		Iteration : 4 ||  x2 = 0.300819 || x1 = 2.32486 \\
		Iteration : 5 ||  x2 = 1.16027 || x1 = 0.300819 \\
		Iteration : 6 ||  x2 = 0.00861099 || x1 = 1.16027 \\
		Iteration : 7 ||  x2 = 0.300819 || x1 = 0.00861099 \\
		Iteration : 8 ||  x2 = 2.12823e-07 || x1 = 0.300819 \\
		
		$x_s=2.12823149138563e-07$
		
		$g(x_s)=x_s-atan(x_s)=3.20284333180532e-21$
	
		x0 =	-5
		
		Iteration : 1 ||  x2 = -3.6266 || x1 = -2.32486 \\
		Iteration : 2 ||  x2 = -1.16027 || x1 = -3.6266 \\
		Iteration : 3 ||  x2 = -2.32486 || x1 = -1.16027 \\
		Iteration : 4 ||  x2 = -0.300819 || x1 = -2.32486 \\
		Iteration : 5 ||  x2 = -1.16027 || x1 = -0.300819 \\
		Iteration : 6 ||  x2 = -0.00861099 || x1 = -1.16027 \\
		Iteration : 7 ||  x2 = -0.300819 || x1 = -0.00861099\\ 
		Iteration : 8 ||  x2 = -2.12823e-07 || x1 = -0.300819 \\
		
		$x_s =	-2.12823149138563e-07$
		$g(x_s)=x_s-atan(x_s)=-3.20284333180532e-21$
		
	
		x0 =	1
		
		Iteration : 1 ||  x2 = 0.214602 || x1 = 0.00320628 \\
		Iteration : 2 ||  x2 = 1.0987e-08 || x1 = 0.214602 \\
		
		$x_s =	1.09870240240853e-08$ \\
		$g(x_s)=x_s-atan(x_s)=0$
		
	
		
		$x0 = -1$
		
		Iteration : 1 ||  x2 = -0.214602 || x1 = -0.00320628 \\
		Iteration : 2 ||  x2 = -1.0987e-08 || x1 = -0.214602 \\
		
	
		
		
		$x0 = 0.1$
		
		
		$x_s = 1.21263429527819e-11$
		$g(x_s)=x_s-atan(x_s)=0$
		
	
		
		}
\end{homeworkSection}

\begin{homeworkSection}{2b}
\problemAnswer{
	The number of iterations reduce as we approach the exact solution.
	
	TODO: Expand
	
	}
\end{homeworkSection}


\end{homeworkSection}

\begin{homeworkSection}{3}
	\begin{homeworkSection}{3a}
	
	\problemAnswer{
		
		$|x_{k+1}-\sqrt{a}| \leq \frac{1}{2}|x_k-\sqrt{a}|$
		
		Extending we get.
		
		$|x_{k+1}-\sqrt{a}| \leq \frac{1}{2}|x_k-\sqrt{a}| \leq \frac{1}{4}|x_{k-1}-\sqrt{a}| ... \leq \frac{1}{2^{k+1}}|x_0-\sqrt{a}|$
		}

		\end{homeworkSection}
		
		\begin{homeworkSection}{3b}
			\problemAnswer{
				
				$x0 = 1.1$
				
				Iteration : 1 ||  x2 = 1.00455 || x1 = 1.00001  \\
				
				$x_s = 1.00454545454545$
				
				
			
				
				$x_0 =	2$
				
				Iteration : 1 ||  x2 = 1.25 || x1 = 1.025  \\
				Iteration : 2 ||  x2 = 1.0003 || x1 = 1.25  \\
				Iteration : 3 ||  x2 = 1.025 || x1 = 1.0003  \\
				Iteration : 4 ||  x2 = 1 || x1 = 1.025  \\
				

				
				$x_s = 1.00000004646115$ \\
				
				
			
				
				$x0 = 5$ \\
				
				Iteration : 1 ||  x2 = 2.6 || x1 = 1.49231  \\
				Iteration : 2 ||  x2 = 1.08121 || x1 = 2.6  \\
				Iteration : 3 ||  x2 = 1.49231 || x1 = 1.08121  \\
				Iteration : 4 ||  x2 = 1.00305 || x1 = 1.49231  \\
				Iteration : 5 ||  x2 = 1.08121 || x1 = 1.00305  \\
				Iteration : 6 ||  x2 = 1 || x1 = 1.08121  \\
				
				
				
				$x_s=1.00000463565079$
				
				
			
				$x0 = 10$
				
				Iteration : 1 ||  x2 = 5.05 || x1 = 2.62401  \\
				Iteration : 2 ||  x2 = 1.50255 || x1 = 5.05  \\
				Iteration : 3 ||  x2 = 2.62401 || x1 = 1.50255  \\
				Iteration : 4 ||  x2 = 1.08404 || x1 = 2.62401  \\
				Iteration : 5 ||  x2 = 1.50255 || x1 = 1.08404  \\
				Iteration : 6 ||  x2 = 1.00326 || x1 = 1.50255  \\
				Iteration : 7 ||  x2 = 1.08404 || x1 = 1.00326  \\
				Iteration : 8 ||  x2 = 1.00001 || x1 = 1.08404  \\
				
				
				
				$x_s=1.00000528956427$ \\
				
				
			
				$x0 = 50$ \\
				
				Iteration : 1 ||  x2 = 25.01 || x1 = 12.525 \\
				Iteration : 2 ||  x2 = 6.30242 || x1 = 25.01   \\
				Iteration : 3 ||  x2 = 12.525 || x1 = 6.30242  \\
				Iteration : 4 ||  x2 = 3.23054 || x1 = 12.525  \\
				Iteration : 5 ||  x2 = 6.30242 || x1 = 3.23054  \\
				Iteration : 6 ||  x2 = 1.77004 || x1 = 6.30242  \\
				Iteration : 7 ||  x2 = 3.23054 || x1 = 1.77004  \\
				Iteration : 8 ||  x2 = 1.1675 || x1 = 3.23054  \\
				Iteration : 9 ||  x2 = 1.77004 || x1 = 1.1675  \\
				Iteration : 10 ||  x2 = 1.01202 || x1 = 1.77004  \\
				
				
				
				$x_s=1.01201564410353$
				
				
			The best convergence is obtained in just one iteration when $x_0=1.1$ which is self-explanatory since the
			exact solution is $1$. As the value of $x_0$ moves away from 1, the number of iterations increase too.
				
				}
			
		\end{homeworkSection}
	
\end{homeworkSection}

\begin{homeworkSection}{4}
\begin{homeworkSection}{1}
\problemAnswer{
	
	}
\end{homeworkSection}
\end{homeworkSection}



\end{document}
