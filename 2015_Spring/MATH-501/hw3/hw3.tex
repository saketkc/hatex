%%%%%%%%%%%%%%%%%%%%%%%%%%%%%%%%%%%%%%%%%
% Structured General Purpose Assignment
% LaTeX Template
%
% This template has been downloaded from:
% http://www.latextemplates.com
%
% Original author:
% Ted Pavlic (http://www.tedpavlic.com)
%
% Note:
% The \lipsum[#] commands throughout this template generate dummy text
% to fill the template out. These commands should all be removed when 
% writing assignment content.
%
%%%%%%%%%%%%%%%%%%%%%%%%%%%%%%%%%%%%%%%%%

%----------------------------------------------------------------------------------------
%	PACKAGES AND OTHER DOCUMENT CONFIGURATIONS
%----------------------------------------------------------------------------------------

\documentclass{article}

\usepackage{fancyhdr} % Required for custom headers
\usepackage{lastpage} % Required to determine the last page for the footer
\usepackage{extramarks} % Required for headers and footers
\usepackage{graphicx} % Required to insert images
\usepackage{latexsym}
\usepackage{mathtools}

\usepackage{lipsum} % Used for inserting dummy 'Lorem ipsum' text into the template

\usepackage{amsmath}
%\usepackage{multline}

% Margins
\topmargin=-0.45in
\evensidemargin=0in
\oddsidemargin=0in
\textwidth=6.5in
\textheight=9.0in
\headsep=0.25in 

\linespread{1.1} % Line spacing

% Set up the header and footer
\pagestyle{fancy}
\lhead{\hmwkAuthorName} % Top left header
\chead{\hmwkClass\ : \hmwkTitle} % Top center header
\rhead{\firstxmark} % Top right header
\lfoot{\lastxmark} % Bottom left footer
\cfoot{} % Bottom center footer
\rfoot{Page\ \thepage\ of\ \pageref{LastPage}} % Bottom right footer
\renewcommand\headrulewidth{0.4pt} % Size of the header rule
\renewcommand\footrulewidth{0.4pt} % Size of the footer rule

\setlength\parindent{0pt} % Removes all indentation from paragraphs

%----------------------------------------------------------------------------------------
%	DOCUMENT STRUCTURE COMMANDS
%	Skip this unless you know what you're doing
%----------------------------------------------------------------------------------------

% Header and footer for when a page split occurs within a problem environment
\newcommand{\enterProblemHeader}[1]{
\nobreak\extramarks{#1}{#1 continued on next page\ldots}\nobreak
\nobreak\extramarks{#1 (continued)}{#1 continued on next page\ldots}\nobreak
}

% Header and footer for when a page split occurs between problem environments
\newcommand{\exitProblemHeader}[1]{
\nobreak\extramarks{#1 (continued)}{#1 continued on next page\ldots}\nobreak
\nobreak\extramarks{#1}{}\nobreak
}

\setcounter{secnumdepth}{0} % Removes default section numbers
\newcounter{homeworkProblemCounter} % Creates a counter to keep track of the number of problems

\newcommand{\homeworkProblemName}{}
\newenvironment{homeworkProblem}[1][Problem \arabic{homeworkProblemCounter}]{ % Makes a new environment called homeworkProblem which takes 1 argument (custom name) but the default is "Problem #"
\stepcounter{homeworkProblemCounter} % Increase counter for number of problems
\renewcommand{\homeworkProblemName}{#1} % Assign \homeworkProblemName the name of the problem
\section{\homeworkProblemName} % Make a section in the document with the custom problem count
\enterProblemHeader{\homeworkProblemName} % Header and footer within the environment
}{
\exitProblemHeader{\homeworkProblemName} % Header and footer after the environment
}

\newcommand{\problemAnswer}[1]{ % Defines the problem answer command with the content as the only argument
\noindent\framebox[\columnwidth][c]{\begin{minipage}{0.98\columnwidth}#1\end{minipage}} % Makes the box around the problem answer and puts the content inside
}

\newcommand{\homeworkSectionName}{}
\newenvironment{homeworkSection}[1]{ % New environment for sections within homework problems, takes 1 argument - the name of the section
\renewcommand{\homeworkSectionName}{#1} % Assign \homeworkSectionName to the name of the section from the environment argument
\subsection{\homeworkSectionName} % Make a subsection with the custom name of the subsection
\enterProblemHeader{\homeworkProblemName\ [\homeworkSectionName]} % Header and footer within the environment
}{
\enterProblemHeader{\homeworkProblemName} % Header and footer after the environment
}
   
%----------------------------------------------------------------------------------------
%	NAME AND CLASS SECTION
%----------------------------------------------------------------------------------------
\DeclarePairedDelimiter\ceil{\lceil}{\rceil}
\DeclarePairedDelimiter\floor{\lfloor}{\rfloor}
\newcommand{\hmwkTitle}{Homework\ \# 1 } % Assignment title
\newcommand{\hmwkDueDate}{Wednesday,\ February \ 11,\ 2015} % Due date
\newcommand{\hmwkClass}{MATH-501} % Course/class
\newcommand{\hmwkClassTime}{11:00am} % Class/lecture time
\newcommand{\hmwkAuthorName}{Saket Choudhary} % Your name
\newcommand{\hmwkAuthorID}{2170058637} % Teacher/lecturer
%----------------------------------------------------------------------------------------
%	TITLE PAGE
%----------------------------------------------------------------------------------------

\title{
\vspace{2in}
\textmd{\textbf{\hmwkClass:\ \hmwkTitle}}\\
\normalsize\vspace{0.1in}\small{Due\ on\ \hmwkDueDate}\\
%\vspace{0.1in}\large{\textit{\hmwkClassTime}}
\vspace{3in}
}

\author{\textbf{\hmwkAuthorName} \\
	\textbf{\hmwkAuthorID}
	}
\date{} % Insert date here if you want it to appear below your name

%----------------------------------------------------------------------------------------

\begin{document}

\maketitle

%----------------------------------------------------------------------------------------
%	TABLE OF CONTENTS
%----------------------------------------------------------------------------------------

%\setcounter{tocdepth}{1} % Uncomment this line if you don't want subsections listed in the ToC

\newpage
\tableofcontents
\newpage




\begin{homeworkSection}{Problem \# 1} % Section within problem
\begin{homeworkProblem}[1a]
	\problemAnswer{
			$sin x = p_0+p_1x$
			Consider $\left. \|sin(x)-p_1x-p+0\| \right|_{2} = \int_{-1}^{1}{(sin(x)-p_1x-p_0)}^2dx$
			
			In order to find, $p_1, p_0$ we consider partial derivatives
			\begin{equation}
				\label{1a1}
				\frac{d}{dp_1}\int_{-1}^{1}{(sin(x)-p_1x-p_0)}^2dx=0
			\end{equation}
			and 
			\begin{equation}
				\label{1a2}
				\frac{d}{dp_0}\int_{-1}^{1}{(sin(x)-p_1x-p_0)}^2dx=0
			\end{equation}
			
			Using Liebnitz's formula in \ref{1a1}:
$
				\int_{-1}^{1}{2(-x)(sin(x)-p_1x-p_0)}dx=0
				\implies \int_{-1}^{1}{x.sin(x)-p_1x^2-p_0x}dx=0
				\implies -x .cos(x) \left.  \right|_{-1}^{1} + \int_{-1}^{1} cos(x) dx - \frac{2p1}{3} = 0
$		
Thus, $p_1 = 3(sin(1)-cos(1))$

Similarly using Leibnitz's ruke  on \ref{1a2}: 
		$\int_{-1}^{1}{2(-1)(sin(x)-p_1x-p_0)}dx=0
		\implies p_0 = 0$ (The first two terms are odd terms and hence integrate to 0)
		
	$p_0$ is also justified since $sin(x=0)=0$
	Hence $sin(x) = 3(sin(1)-cos(1))x$
		}
		
\end{homeworkProblem}
\begin{homeworkProblem}[1b]
	\problemAnswer{ Taylor approximation(degree 3) around  $t=0$:
		$p_2(t) = sin(0) + \frac{cos(0)}{1!} {(x-0)^1}  + \frac{-sin(0)(x-0)^2}{2!} + \frac{-cos(0)(x-0)^3}{3!} + R_4$
		
		$p_2(t) = t - \frac{t^3}{3!} + R_4(t)$ where $R_4$ is $o(t^4)$ remainer term.
		}
\end{homeworkProblem}

\begin{homeworkProblem}[1c]
	\problemAnswer{ Given $f(t)$ at $t=-1,\frac{-1}{3}, \frac{1}{3},1$ we fit a degree$3$ polynomial for $sin(x)$ using Legendre Polynomials.
		
		$sin(-1) = - sin(1)$ and $sin(\frac{-1}{3}) = - sin(\frac{1}{3})$\\
		
		Now, 
		$l_0(x) = \frac{(x+\frac{1}{3})(x-\frac{1}{3})(x-1)}{(\frac{-2}{3})(\frac{-4}{3})(-2)}
			= \frac{-9}{16}(x+\frac{1}{3})(x-\frac{1}{3})(x-1)
		$ \\
		
		$l_1(x) = \frac{(x+1)(x-\frac{1}{3})(x-1)}{(\frac{2}{3})(\frac{-2}{3})(\frac{-4}{3})}
		= \frac{27}{16}(x+1)(x-\frac{1}{3})(x-1)
		$ \\
		
		$l_2(x) = \frac{(x+1)(x+\frac{1}{3})(x-1)}{(\frac{4}{3})(\frac{2}{3}))(\frac{-2}{3})}
		= \frac{-27}{16}(x+1)(x+\frac{1}{3})(x-1)
		$ \\
		
		$l_3(x) = \frac{(x+1)(x+\frac{1}{3})(x-\frac{1}{3})}{(2)(\frac{4}{3})(\frac{2}{3}))}
		= \frac{9}{16}(x+1)(x+\frac{1}{3})(x-\frac{1}{3})
		$ \\
		
		$sin(x) = \sum_{i=0}^{3}t_i x l_i(x)$ \\
		
		Thus $sin(x) = (-sin(1))l_0(x) +(-sin(\frac{1}{3}))l_1(x) + sin(\frac{1}{3})l_2(x) + sin(1)l_3(x)$
	}
\end{homeworkProblem}
\end{homeworkSection}

\begin{homeworkSection}{2}
\problemAnswer{
	Given: $u_1=1, u_2 = x, u_3 = x^2$
	$w_i(x) = \frac{v_i}{\| v_i \|}$ where $v_i$ is given by:
	$ v_1 = u_1 = 1$ \\
	$ v_2 = u_2 - \frac{\langle u_2, v_1 \rangle v_1}{\langle v_1, v_1 \rangle}$
	$ v_3 = u_3 - \frac{\langle u_3, v_1 \rangle v_1}{\langle v_1, v_1 \rangle} -\frac{\langle u_3, v_2 \rangle v_2}{\langle v_2, v_2 \rangle}$
	
	
	$v_2 = x - \frac{\int_{0}^{1} x dx}{\int_0^{1} 1^2 dx}
		= x - \frac{1}{2}$
		
	$v_3 = x^2- (\int_0^1 x^2 dx + (x-\frac{1}{2}) \frac{\int_0^1 x^2 (x- \frac{1}{2})dx}{\int_0^1 (x-\frac{1}{2})^2 dx}
	  = x^2 -x + \frac{1}{6}
		$
		
	Similarly,
	$w_1 = 1$\\
	$w_2 =  \frac{x-\frac{1}{2}}{\sqrt{\frac{1}{12}}}$ \\
	$w_3 = \frac{x^2 -x + \frac{1}{6}}{\sqrt{\frac{1}{180}} \\ }$
	
	$Pf = \sum_{i=1}^{3}\langle f, w_i \rangle w_i
	= (\int_0^1 \sqrt{x} dx)1 + (\int_0^1 \sqrt{x}(x-\frac{1}{2})dx) (x-\frac{1}{2}) + (\int_0^1 \sqrt{x} (x^2-x+\frac{1}{6})dx) (x^2-x+\frac{1}{6})$
	
	
	$ Pf = \frac{2}{3} + \frac{4}{5} (x-\frac{1}{2}) + \frac{-4}{7}  (x^2-x+\frac{1}{6})$
	}
\end{homeworkSection}

\begin{homeworkSection}{3}
	\problemAnswer{
		By Weierstrass' approxmiation theorem for $\epsilon > 0$ there exists a polynomial $p(x)$ such that 
		$\| \left. p-f \right \|_{\infty} = max|f(x)-p(x)| < \epsilon$ while $ a \leq x \leq b$
		$|E_n(f)| = |E_n(f) - E_n(p)|\\
		= |\int_0^1 f(x) dx - \sum_{i=1}^{n}w_i f(x_i)|\\
		= |\int_0^1 f(x) dx - \int_0^1 p(x) dx + \sum_{i=1}^{n}w_i p(x_i) - \sum_{i=1}^{n}w_i f(x_i)|\\
		= |\int_0^1 (f(x) - p(x)) dx + \sum_{i=1}^{n}w_i (p(x_i) - f(x_i))|$\\
		Thus,\\
		$|E_n(f)| = |\int_0^1 (f(x) - p(x)) dx + \sum_{i=1}^{n}w_i (p(x_i) - f(x_i))|$ (Since, $E_n(P)=0$
		
		Applying triangular inequality,\\
		$E_n(f) \leq = \int_0^1 |(f(x) - p(x))| dx + \sum_{i=1}^{n}w_i |(p(x_i) - f(x_i))|
		\approx\| \left. p-f \right \|_{\infty} + \| \left. p-f \right \|_{\infty} 
		\leq \epsilon $\\
		Thus,\\ 
		
		$\| \left. p-f \right \|_{\infty} \leq \frac{\epsilon}{2}$ and hence there exists a $N > 0$ such that $|E_n(f)| < \epsilon$  when $n > N$
		
		
		}
\end{homeworkSection}
\end{document}
