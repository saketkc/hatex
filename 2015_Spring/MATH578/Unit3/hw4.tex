\documentclass{article}
%\usepackage[margin=0.5in]{geometry}

\begin{document}
	
\title{HW\#4 || Graph Clustering Algorithms}
\author{Saket Choudhary}
\maketitle

\section{Markov Clustering}

Markov clustering makes use of stochastic flows in a graph. The underlying idea is
that a random walk on a dense cluster is likely to stay on the vertices of this cluster before
jumping to other cluster. This property is made use by simulating a stochastic flow on the graph
such that the flow be promoted wherever the current is strong and downweighted otherwise. This can 
thus be formulated as a Markcov graph. The flows are modelled by calculating successive powers of the matrix.

\section{Restricted Neighborhood Search}
RNS clustering tries to minimises a cost function that captures the weight of
inter and intra cluster edges. With a random initial value, RNSC iteratively assigns
a node to other cluster if that leads to a local minima. Termination can be defined based on number of iterations
or convergence of the cost function.

\section{Molecular Complex Detection}
Molecular complex detection assigns a weight to each vertex that is proportional
to the number of neighbors. Starting from the heaviest vertex it iteratively moves out
assigning each vertex to the cluster if it is above a certain threshold.
\section{Comparison}
\end{document}