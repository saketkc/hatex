\documentclass[a4paper,11pt,x11names]{article}

\usepackage{hyperref}

\newcommand{\hmwkTitle}{Assignment\ \# 4 } % Assignment title
\newcommand{\hmwkDueDate}{Monday,\ July \ 27,\ 2015} % Due date
\newcommand{\hmwkClass}{CSCI-585} % Course/class
\newcommand{\hmwkClassTime}{11:00pm} % Class/lecture time
\newcommand{\hmwkAuthorName}{Saket Choudhary} % Your name
\newcommand{\hmwkAuthorID}{2170058637} % Teacher/lecturer
%----------------------------------------------------------------------------------------
%	TITLE PAGE
%----------------------------------------------------------------------------------------

\title{
\vspace{2in}
\textmd{\textbf{\hmwkClass:\ \hmwkTitle}}\\
\normalsize\vspace{0.1in}\small{Due\ on\ \hmwkDueDate}\\
%\vspace{0.1in}\large{\textit{\hmwkClassTime}}
\vspace{3in}
}

\author{\textbf{\hmwkAuthorName} \\
	\textbf{\hmwkAuthorID}
	}
\date{} % Insert date here if you want it to appear below your name


\begin{document}
\maketitle
\clearpage

\section{Question \# 1} % Section within problem
\subsection{1a}
To enforce the primiary key constraint we can use an aritificial primairy key, an 'ID'
which is auto incremented.

\subsection{1b}

Query that is fdefinitely faster with an indexed gpa: 'SELECT studentID FROM Student WHERE Student.gpa = 3';

\subsection{1c}

Any INSERT operation if indexing is on gpa.


\subsection{1d}<`0`>

Any SELECT operation when the query when the index is based on  majorID is uncertain, since majorID
is supposed to have high sparsity. And hence any access will require same number of row accesses
irrespective of whether majorID is indexed or not.

\section{Question \#2}

\subsection{1a}
(v) No index. Since the query is going to access all the records anyway, there is little use of an index 
specific for this query.

\subsection{1b}
(iv) Clustered B+ tree index on (schoolID, budget) allows scan in order, so the search can stop at school 3
followed by all budget lesser than \$150000

\section{Question \# 3}


\end{document}


