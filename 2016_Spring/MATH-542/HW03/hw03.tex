\documentclass[a4paper]{article}

\usepackage[english]{babel}
\usepackage[utf8x]{inputenc}
\usepackage{amsmath}
\usepackage{graphicx}
\usepackage[colorinlistoftodos]{todonotes}

\title{MATH 542 Homework 3}
\author{Saket Choudhary\\skchoudh@usc.edu}

\begin{document}
\maketitle 
\section*{Problem 1}

Given: $A_{n\times n}$ is idempotent $\implies AA=A$; $P$ is non-singular and $C_{n\times n}$ is orthogonal $\implies CC^T=C^TC=I$

\subsection*{1.a}
\begin{align*}
(I-A)(I-A) = I-A-A+AA\\
&= I-2A+A\ \text{ using } AA=A\\
&=I-A
\end{align*}

Hence $I-A$ is idempotent

\subsection*{1.b}
\begin{align*}
A(I-A) &= A-AA\\
&= A-A\ \text{ using } AA=A\\
&=0
\end{align*}

Similarly,
\begin{align*}
(I-A)A &= A-AA\\
&= A-A\ \text{ using } AA=A\\
&=0
\end{align*}

\subsection*{1.c}
\begin{align*}
(P^{-1}AP)(P^{-1}AP)  &=P^{-1}APP^{-1}AP\\ 
&= P^{-1}AAP \ \text{since }PP^{-1}=I\\
&= P^{-1}AP\ \text{ using } AA=A\\
\end{align*}
Hence $P^{-1}AP$ is idempotent

\subsection*{1.d}
\begin{align*}
(C'AC)(C'AC) &=C'ACC'AC\\
&= C'AAC \text{ since } CC'=C'C=I\\
&= C'AC \ \text{ using } AA=A\\
\end{align*}

\subsection*{1.e}
$A$ is a projection matrix and $C'C=C'C=I$
For $C'AC$ to be a projection matrix $C'ACz=z\ \forall z \in S$ for some vector space $S$

For some $z \in S$:
\begin{align*}
C'ACz &= C'Ay \text{ where } y=Az\\
&= C'Ay\\
&= C'y\text{ since $A$ is projection matrix }  \\
&= C'Cz\\
&= z\ \text{ since } C'C=I
\end{align*}

Thus, $C'ACz \in S$ and $C'ACz=z$ and hence $C'AC$ is a projection matrix

\section*{Problem 2}
\subsection*{2.a}

\begin{align*}
Q(x_1,x_2,x_3) &= 12x_1^2+3x_2^2+3x_3^2+2x_1x_2-10x_1x_3+4x_2x_3\\
&= 3(2x_1+x_2+x_3)^2-(12x_1x_2+6x_2x_3+12x_1x_3)+2x_1x_2-10x_1x_3+4x_2x_3\\
&= 3(2x_1+x_2+x_3)^2-10x_1x_2-2x_2x_3-22x_1x_3\\
&= (x_1+x_2)^2+ 2(x_2+x_3)^2+(5x_1-x_3)^2-14x_1^2\\
&=(x_1+x_2)^2+2(x_2+x_3)^2+x_3^2+11x_1^2-10x_1x_3\\
&= (x_1+x_2)^2+2(x_2+x_3)^2+(5x_1-x_3)^2-14x_1^2
\end{align*}

and hence $Q(x_1,x_2,x_3)$ is neither positive definite nor positive semidefinite.

\subsection*{2.b}
\begin{align*}
x'Ax=\begin{pmatrix}x_1 & x_2 &x_3\end{pmatrix}\begin{pmatrix}
2 & 1 & 2\\
1 & 2 & 1\\
2 & 1 & 4\\
\end{pmatrix}\begin{pmatrix} x_1\\x_2\\x_3
\end{pmatrix}\\
&= (2x_1+x_2+x_3)x_1+(x_1+2x_2+x_3)x_2+(2x_1+x_2+4x_3)x_3\\
&= 2x_1^2+2x_2^2+4x_3^2+2x_1x_2+3x_1x_3+2x_2x_3\\
&= (x_1+x_2)^2+(x_2+x_3)^2+(x_1+\frac{3}{2}x_2)^2+\frac{3}{4}x_2^2\\
\geq 0
\end{align*}

$A$ is positive semidefinite

\subsection*{2.c}

\begin{align*}
x'Ax=\begin{pmatrix}x_1 & x_2 &x_3\end{pmatrix}\begin{pmatrix}
1 & 2 & 3\\
2 & 1 & 1\\
3 & 1 & -2\\
\end{pmatrix}\begin{pmatrix} x_1\\x_2\\x_3
\end{pmatrix}\\
&= (x_1+2x_2+3x_3)x_1+(2x_1+x_2+x_3)x_2+(3x_1+x_2-2x_3)x_3\\
&= x_1^2+x_2^2-2x_3^2+4x_1x_2+6x_1x_3+2x_2x_3\\
&= (x_2+x_3)^2+3(x_1+x_3)^2+2(x_1+x_2)^2-6x_1^2-6x_3^2-4x_1^2-2x_2^2-6x_3^2\\
\end{align*}

$A$ is neither positive definite nor positive semidefinite.

\section*{Problem 3}

\begin{align*}
\begin{pmatrix}
a & b\\
c & d\\
\end{pmatrix}\times \begin{pmatrix}
a & b\\
c & d\\
\end{pmatrix} &= \begin{pmatrix}
2 & -1\\
-1 & 2
\end{pmatrix}\\
\begin{pmatrix}
a^2+b^2 & ac+bd\\
ac+bd & c^2+d^2
\end{pmatrix}&=\begin{pmatrix}
2 & -1\\
-1 & 2
\end{pmatrix}\\
\end{align*}

\begin{align*}
a^2+b^2 &=2\\
ac+bd&=-1\\
c^2+d^2&=2
\end{align*}

Let $a=1,b=1$ $\implies \ c+d=1 $ 

\begin{align*}
c^2+d^2 &=2\\
c^2+(-1-c)^2 & =2\\
2c^2+2c-1 &=0\\
c &= \frac{-2\pm\sqrt{4+8}}{4}\\
c &= \frac{-1\pm\sqrt{3}}{2}\\
d &= \frac{-1\pm\sqrt{3}}{2}
\end{align*}

and hence one possible $B$ is:
\begin{align*}
B &=\begin{pmatrix}
1 & 1\\
\frac{-1-\sqrt{3}}{2} & \frac{-1+\sqrt{3}}{2}\\
\end{pmatrix}
\end{align*}

\section*{Problem 4}
$G=(X'X)^-$

Let $Y=(X'X)$ so that $G=Y^-$

\begin{align*}
G(X'X)G &= GYG\\
&= Y^-YY^-\\
&= Y^-
\end{align*}

Note that $Y$ is symmetric, $Y=Y^T$
(Proof: $Y=X^TX$, $Y^T=(X^TX)^T = X^TX$)

Since $Y$ is symmetric, $Y^-$ is symmetric too $\implies G(X'X)G$ is symmetric.

To prove $P=G(X'X)G$ is a generalised inverse of $Y=X'X$ we need to show $YPY=P$

\begin{align*}
YPY &= YGYGY\\
&= Y(Y^-)Y\ \text{using $GYG=Y^-$ from the previous result} \\
&= Y \ \text{ since } YY^-=I
\end{align*}

\section*{Problem 5}

Given: $E[X]=1$ and $Var(X)=E[X^2]-E[X]^2 = 5$
Using $E[X]=1$ we get, $E[X^2]=5+E[X]^2=6$

\subsection*{5.a}

\begin{align*}
E[(1+2X)^2] &= E[1+4X+4X^2]\\
&= E[1]+E[4X]+E[4X^2]\\
&= 1+4E[X]+4E[X^2]\\
&= 1+4+4(6)\\
&= 30
\end{align*}

\subsection*{5.b}
\begin{align*}
var(3+4X) &= Var(3) + Var(4X) + 2Cov(3,4X)\\
&= 0 + 16Var(X) + 2(0)\\
&= 80
\end{align*}

\end{document}