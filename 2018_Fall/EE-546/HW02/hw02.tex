%%%%%%%%%%%%%%%%%%%%%%%%%%%%%%%%%%%%%%%%
% Structured General Purpose Assignment
% LaTeX Template
%
% This template has been downloaded from:
% http://www.latextemplates.com
%
% Original author:
% Ted Pavlic (http://www.tedpavlic.com)
%
% Note:
% The \lipsum[#] commands throughout this template generate dummy text
% to fill the template out. These commands should all be removed when 
% writing assignment content.
%
%%%%%%%%%%%%%%%%%%%%%%%%%%%%%%%%%%%%%%%%%

%----------------------------------------------------------------------------------------
%       PACKAGES AND OTHER DOCUMENT CONFIGURATIONS
%----------------------------------------------------------------------------------------

\documentclass{article}

\usepackage{fancyhdr} % Required for custom headers
\usepackage{lastpage} % Required to determine the last page for the footer
\usepackage{extramarks} % Required for headers and footers
\usepackage{graphicx} % Required to insert images
\usepackage{lipsum} % Used for inserting dummy 'Lorem ipsum' text into the template
\usepackage{url,amsmath,amsfonts,amssymb, bm}


% Margins
\topmargin=-0.45in
\evensidemargin=0in
\oddsidemargin=0in
\textwidth=6.5in
\textheight=9.0in
\headsep=0.25in 

\linespread{1.1} % Line spacing

% Set up the header and footer
\pagestyle{fancy}
\lhead{\hmwkAuthorName} % Top left header
\chead{\hmwkClass\ : \hmwkTitle} % Top center header
\rhead{\firstxmark} % Top right header
\lfoot{\lastxmark} % Bottom left footer
\cfoot{} % Bottom center footer
\rfoot{Page\ \thepage\ of\ \pageref{LastPage}} % Bottom right footer
\renewcommand\headrulewidth{0.4pt} % Size of the header rule
\renewcommand\footrulewidth{0.4pt} % Size of the footer rule

\setlength\parindent{0pt} % Removes all indentation from paragraphs

\newtheorem{definition}{Definition}
\newtheorem{remark}{Remark}
\newtheorem{properties}{Properties}
\newtheorem{example}{Example}
\newtheorem{theorem}{Theorem}
\newtheorem{lemma}[theorem]{Lemma}
\newtheorem{corollary}[theorem]{Corollary}
\newtheorem{proposition}[theorem]{Proposition}
\newtheorem{claim}[theorem]{Claim}
\newtheorem{observation}{Observation}

\def \endprf{\hfill {\vrule height6pt width6pt depth0pt}\medskip}
\newenvironment{proof}{\noindent {\bf Proof} }{\endprf\par}

\newcommand{\R}{{\mathbb R}}
\newcommand{\Z}{{\mathbb Z}}
\newcommand{\Q}{{\mathbb Q}}
\newcommand{\C}{{\mathbb C}}
\newcommand{\N}{{\mathbb N}}


\newcommand{\E}{{\mathbb E}}

\newcommand{\A}{\mathcal{A}}
\newcommand{\vct}[1]{\bm{#1}}
\newcommand{\mtx}[1]{\bm{#1}}
%----------------------------------------------------------------------------------------
%       DOCUMENT STRUCTURE COMMANDS
%       Skip this unless you know what you're doing
%----------------------------------------------------------------------------------------

% Header and footer for when a page split occurs within a problem environment
\newcommand{\enterProblemHeader}[1]{
\nobreak\extramarks{#1}{#1 continued on next page\ldots}\nobreak
\nobreak\extramarks{#1 (continued)}{#1 continued on next page\ldots}\nobreak
}

% Header and footer for when a page split occurs between problem environments
\newcommand{\exitProblemHeader}[1]{
\nobreak\extramarks{#1 (continued)}{#1 continued on next page\ldots}\nobreak
\nobreak\extramarks{#1}{}\nobreak
}

\setcounter{secnumdepth}{0} % Removes default section numbers
\newcounter{homeworkProblemCounter} % Creates a counter to keep track of the number of problems

\newcommand{\homeworkProblemName}{}
\newenvironment{homeworkProblem}[1][Problem \arabic{homeworkProblemCounter}]{ % Makes a new environment called homeworkProblem which takes 1 argument (custom name) but the default is "Problem #"
\stepcounter{homeworkProblemCounter} % Increase counter for number of problems
\renewcommand{\homeworkProblemName}{#1} % Assign \homeworkProblemName the name of the problem
\section{\homeworkProblemName} % Make a section in the document with the custom problem count
\enterProblemHeader{\homeworkProblemName} % Header and footer within the environment
}{
\exitProblemHeader{\homeworkProblemName} % Header and footer after the environment
}

\newcommand{\problemAnswer}[1]{ % Defines the problem answer command with the content as the only argument
\noindent\framebox[\columnwidth][c]{\begin{minipage}{0.98\columnwidth}#1\end{minipage}} % Makes the box around the problem answer and puts the content inside
}

\newcommand{\homeworkSectionName}{}
\newenvironment{homeworkSection}[1]{ % New environment for sections within homework problems, takes 1 argument - the name of the section
\renewcommand{\homeworkSectionName}{#1} % Assign \homeworkSectionName to the name of the section from the environment argument
\subsection{\homeworkSectionName} % Make a subsection with the custom name of the subsection
\enterProblemHeader{\homeworkProblemName\ [\homeworkSectionName]} % Header and footer within the environment
}{
\enterProblemHeader{\homeworkProblemName} % Header and footer after the environment
}
   
%----------------------------------------------------------------------------------------
%       NAME AND CLASS SECTION
%----------------------------------------------------------------------------------------

\newcommand{\hmwkTitle}{Assignment\ \# 2} % Assignment title
%\newcommand{\hmwkDueDate}{} % Due date
\newcommand{\hmwkClass}{EE-546} % Course/class
\newcommand{\hmwkClassTime}{} % Class/lecture time
\newcommand{\hmwkAuthorName}{Saket Choudhary} % Your name
\newcommand{\hmwkAuthorID}{2170058637} % Teacher/lecturer
%----------------------------------------------------------------------------------------
%       TITLE PAGE
%----------------------------------------------------------------------------------------

\title{
\vspace{2in}
\textmd{\textbf{\hmwkClass:\ \hmwkTitle}}\\
\normalsize\vspace{0.1in}%\small{Due\ on\ \hmwkDueDate}\\
\vspace{0.1in}\large{\textit{\hmwkClassTime}}
\vspace{3in}
}

\author{\textbf{\hmwkAuthorName} \\
        \textbf{\hmwkAuthorID}
        }
\date{} % Insert date here if you want it to appear below your name

%----------------------------------------------------------------------------------------

\begin{document}

\maketitle

%----------------------------------------------------------------------------------------
%       TABLE OF CONTENTS
%----------------------------------------------------------------------------------------

%\setcounter{tocdepth}{1} % Uncomment this line if you don't want subsections listed in the ToC

\newpage
\tableofcontents
\newpage


%----------------------------------------------------------------------------------------
%       PROBLEM 2
%----------------------------------------------------------------------------------------

\begin{homeworkProblem}[Problem \# 1] % Custom section title


%\begin{homeworkSection}{(3)} % Section within problem
%\lipsum[4]\vspace{10pt} % Question

\problemAnswer{ % Answer
Problem 1a):
\begin{align*}
    Z &= \frac{1}{m} \sum_{r=1}^m X_r |X_r| \text{ sign}(X_r + \mathbf{a_r}^T \mathbf{y})\\
    &= \frac{1}{m} \sum_{r=1}^m X_r^2 \text{ sign}(X_r) \text{sign}(X_r + \mathbf{a_r}^T \mathbf{y})\\
    \text{Let } Z_r &= X_r^2 \text{ sign}(X_r) \text{sign}(X_r + \mathbf{a_r}^T \mathbf{y})\\
    \mathbf{a_r}^T\mathbf{y} &\sim \mathcal{N}(0, \mathbf{y^T}\mathbf{y})\\
    \implies \frac{\mathbf{a_r}^T\mathbf{y}}{\sqrt{\mathbf{y^T}\mathbf{y}}} &\sim \mathcal{N}(0, 1)\\
    \mathbb{E}[Z_r] &= \mathbb{E}[X_r^2 \text{sign}(X_r) \text{sign}(X_r+\mathbf{a_r}^T\mathbf{y})] \\
    &= \mathbb{E}[X_r^2 \text{sign}(X_r) \text{sign}(X_r+\sqrt{\mathbf{y^T}\mathbf{y}}\frac{\mathbf{a_r}^T\mathbf{y}}{\sqrt{\mathbf{y^T}\mathbf{y}}})] \\
    &= \frac{2}{\pi} \tan^{-1} \frac{1}{\sqrt{\mathbf{y^T}\mathbf{y}}} + \frac{2}{\pi} \frac{\sqrt{\mathbf{y^T}\mathbf{y}}}{1+\mathbf{y^T}\mathbf{y}}\\
    \implies \mathbb{E}[Z] &= \frac{1}{m}\sum_{r=1}^m\mathbb{E}[Z_r]\\
    &= \frac{2}{\pi} \tan^{-1} \frac{1}{\sqrt{\mathbf{y^T}\mathbf{y}}} + \frac{2}{\pi} \frac{\sqrt{\mathbf{y^T}\mathbf{y}}}{1+\mathbf{y^T}\mathbf{y}}\\
\end{align*}

Problem 1b):


\begin{align*}
\mathbb{P}(|X_r^2 \text{ sign}(X_r) \text{sign}(X_r + \mathbf{a_r}^T \mathbf{y})| \geq t) &= \mathbb{P}(|X_r^2 | \geq t) & \because \text{sign is immaterial}\\
&= P(\chi_1^2 \geq t)
\end{align*}
$P(\chi^2 \geq t)$ is easily obtained as $\chi^2$ is sub-exponential,

$Z \sim \chi_m^2$ and hence:
\begin{align*}
    P(|Z - \mathbb{E}[Z]| \geq t) &\leq \exp({1-\frac{t}{k_1}})
\end{align*}
}
%\end{homeworkSection}

%--------------------------------------------


%--------------------------------------------

\end{homeworkProblem}

%----------------------------------------------------------------------------------------
%       PROBLEM 3
%----------------------------------------------------------------------------------------
\begin{homeworkProblem}[Problem \# 2] % Custom section title


%\begin{homeworkSection}{(3)} % Section within problem
%\lipsum[4]\vspace{10pt} % Question

\problemAnswer{
Problem 2 (i):

From Cauchy Schwartz inequality we have:
\begin{align*}
    |\langle x, y \rangle| &\leq ||x|| ||y||
\end{align*}

Thus,

\begin{align*}
    \langle x, Ay \rangle &\leq ||x||\ ||Ay|| \\
    \max_{||x||_{l_2} =1, ||y||_{l_2}=1}\langle x, Ay \rangle &\leq \max_{||x||_{l_2} =1, ||y||_{l_2}=1}||x||\ ||Ay|| \\
    &= \max_{||y||_{l_2}=1}||Ay|| &  \because ||x||=1\\
    &= ||A|| & \because \text{from the definition of } ||A||
\end{align*}
Fact: For unitary $U$, $U^TU=I$ $\implies$  $||Ux||_{l_2}^2 = x^TU^TUx = ||x||_{l_2}^2$

Now consider $||Ax||$:
\begin{align*}
    \sup_{||x||=1}||Ax|| &= \sup_{||x||=1} ||U\Sigma V^T x|| & \because \text{ spectral decomosition of A} \\
    &= \sup_{||x||=1} ||\Sigma V^T x|| & \because ||Ux|| = ||x|| for unitary U \\
    &= \sup_{||y||=1} ||\Sigma y|| & \because y=V^Tx \text{ and } y^Ty = 1\\
\end{align*}

Since $\Sigma$ is a diagnoal matrix  $\sup_{||y||=1} ||\Sigma y||$ can easily be obtained when $y = (1,0, \dots,0)$ such that the maximum will be $\sigma_1(A)$

Problem 2 (ii):

\begin{align*}
    \text{trace}(U^TAV) &= \sum_{i=1}^r u_i^Ta_{ii}v_i \\
    \max \text{trace}(U^TAV) &= \max \sum_{i=1}^r u_i^Ta_{ii}v_i \\
    &= \sum_{i=1}^r \sigma_i(A)
\end{align*}

}
%\end{homeworkSection}

%--------------------------------------------


%--------------------------------------------

\end{homeworkProblem}

\newpage
\begin{homeworkProblem}[Problem \# 3] % Custom section title
\problemAnswer{
Upper bound:

\begin{align*}
||A|| &= \sup_{||x||=1} ||Ax|| = \sup_{||x||=1} \sqrt{\sum_{i=1}^m\big(\sum_{j=1}^n A_{ij}x_j \big)^2} \\
&\leq \sup_{||x||=1} \sqrt{\sum_{i=1}^m ( \sum_{j=1}^nA_{ij}^2) ( \sum_{k=1}^nx_k^2 )} & \because \text{ using Cauchy-Schwartz inequality }\\
\implies ||A|| &\leq \sqrt{m}\max_{i \in \{1, 2, \dots, m\}}  \big( \sum_{j=1}^n A_{ij}^2 \big)^{1/2}
\end{align*}
For eqaulity $A_{ij}x_j = \lambda x_j$ and hence $A = I$ is one such matrix.

Lower bound:

Consider $x = (\frac{1}{\sqrt{n}} , \frac{1}{\sqrt{n}} \dots, \frac{1}{\sqrt{n}})$

them 
\begin{align*}
    ||A|| = \sup_{||x||=1}||Ax||  & \geq \sqrt{\sum_{i=1}^m\big(\sum_{j=1}^n A_{ij}\frac{1}{\sqrt{n}} \big)^2} \\
    &= \frac{1}{\sqrt{n}} \sqrt{\sum_{i=1}^m\big(\sum_{j=1}^n A_{ij} \times 1 \big)^2} \\
    &\geq \frac{1}{\sqrt{n}} \sqrt{\frac{1}{m} \big ( \sum_{i=1}^m | \sum_{j=1}^n A_{ij}| \big)^2}   \\
    &= \frac{1}{\sqrt{mn}}  \sum_{i=1}^m | \sum_{j=1}^n A_{ij}|
\end{align*}

}

\end{homeworkProblem}

\begin{homeworkProblem}[Problem \# 4] % Custom section title
\problemAnswer{
\begin{align*}
    ||A||_F &= (\sum_{i=1}^m \sum_{j=1} ^n |a_{ij}|^2)^\frac{1}{2}\\
    &= (\text{Tr}(A^TA))^\frac{1}{2}\\
    &= (\text{Tr}(V\Sigma U^T U \Sigma V^T ))^\frac{1}{2} & \because \text{Spectral decomposition of } A\\ 
    &= (\text{Tr}(V\Sigma^2 V^T ))^\frac{1}{2}\\
    &= (\text{Tr}(V^TV\Sigma^2  ))^\frac{1}{2} & \because \text{Tr}(AB) = \text{Tr}(BA)\\
    &= (\text{Tr}(\Sigma^2  ))^\frac{1}{2} \\
    &= (\sum_{i=1}^{\min(m,n)} \sigma_i^2(A) )^\frac{1}{2} \\
\end{align*}

From Problem 2 $||A|| = \sigma_1(A)$. Now, given $\sigma_1(A) \geq \sigma_2(A) \geq \dots \geq \sigma_{\min(m, n)}(A)$

\begin{align*}
    \sigma_1(A) &\leq \sqrt{ \sigma_1(A)^2  + \sigma_2(A)^2 + \dots + \sigma_{\min(m,n)}(A)^2 }\\
    \sqrt{ \sigma_1(A)^2  + \sigma_2(A)^2 + \dots + \sigma_{\min(m,n)}(A)^2 } &\leq \sqrt{ \sigma_1(A)^2  + \sigma_1(A)^2 + \dots + \sigma_1(A)^2 }\\
    &= \sqrt{\text{rank}(A)}\sigma_1(A)
\end{align*}
where the l;ast equality follows from the fact that  only  rank($A$) singular vectors are non zero.

Thus,

$$
  \sigma_1(A) \leq \sqrt{ \sigma_1(A)^2  + \sigma_2(A)^2 + \dots + \sigma_{\min(m,n)}(A)^2 } \leq \sqrt{\text{rank}(A)}\sigma_1(A)
$$

$\implies$ $$ \sigma_1(A) \leq ||A|| \leq \sqrt{\text{rank}(A)}||A||$$
}

\end{homeworkProblem}

\end{document}
