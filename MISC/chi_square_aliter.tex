\documentclass{article}
\usepackage{latexsym}
\usepackage{mathtools}
\usepackage{amsmath}
\begin{document}
\begin{equation}
\begin{split}
	\sum {(X_i-\bar{X})^2} = \sum {(X_i-\mu + \mu-\bar{X} )^2} \\
	= \sum(X_i-\mu)^2+\sum(\mu-\bar{X})^2 + 2\sum{(X_i-\mu)(\mu-\bar{X})} \\
	= A + B + C	
	\end{split}
\end{equation}
\linebreak\linebreak\linebreak\linebreak
\begin{equation}
\begin{split}
	C = 2(\sum{X_i\mu -X_i\bar{X} - \mu^2 - \mu \bar{X} }) \\
	= 2(n\bar{X}\mu -n\bar{X}^2 -n\mu^2 - n\mu\bar{X})\\
	=-2n(\mu^2+\bar{X}^2)
		\end{split}
\end{equation}

\begin{equation}
	B+C=n(\mu^2+\bar{X}^2-2\mu\bar{X})-2n(\mu^2+\bar{X}^2)\\
	=-n(\mu-\bar{X})^2
\end{equation}

\begin{align*}
\sum {(X_i-\bar{X})^2} = \sum(X_i-\mu)^2 -n(\mu-\bar{X})^2 \\
\frac{\sum {(X_i-\bar{X})^2}}{\sigma^2} = \frac{\sum(X_i-\mu)^2}{\sigma^2} -\frac{n(\mu-\bar{X})^2}{\sigma^2}\\
Q=R-T\\
R=Q+T
\end{align*}
Now, $R \sim \chi^2(n)$ and $T=\chi^2(1)$ and 
$M_{\chi^2(n)}(t)=\frac{1}{(1-2t)^{\frac{n}{2}}}$

Using independence of Q and T(proved in theorem's part a in class) we have
$M_R=M_Q.M_T$ and so $M_Q=\frac{M_R}{M_T}$ giving:
$M_Q=\frac{(1-2n)^{\frac{-n}{2}}} {{(1-2n)}^{\frac{-1}{2}}} = (1-2n)^{\frac{-(n-1)}{2}}$
Thus $Q=\frac{(n-1)S^2}{\sigma^2} \sim \chi^2(n-1)$
\end{document}